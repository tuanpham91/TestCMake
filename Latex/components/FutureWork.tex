\chapter{Conclusion}

The results in the previous section show that control of the robot via the augmented Jacobian matrix is a valid approach to solving the inverse kinematics problem. When performing sufficiently small steps the deviation of the RCM from its ideal position was small enough as to not put stress on the tissue surrounding the incision site. Even though the method proved to be precise enough in simulation, the precision for the real robot would most likely be lower since no dynamic effects were considered. To improve the overall accuracy, there are several points that can be relatively easily implemented building on the results presented but exceed the scope of this thesis. These include: 

\paragraph{Preventing drift of the RCM position:} As discussed in Section \ref{Acc_PI}, the x-position of the RCM point experiences some drift during the course of the injection. While the drift is relatively small compared to the size of the trocar, a proper solution might significantly increase the performance. Within each iteration step, the RCM is forced to the target position ($x_{rcm}, y_{rcm}, z_{rcm}$), allowing a small deviation from the target position as a trade-off between drift rate and computation time. By adjusting the corresponding entry in the matrix $K$, the error of $x_{rcm}$ is given a higher priority and will drift less, but all other position parameters will converge slower leading to a worse overall performance. This problem might be easily resolved by readjusting the position of the RCM on the needle in each iteration step before enforcing the normal control loop. This means that before each iteration the point on the needle closest to ($x_{rcm}, y_{rcm}, z_{rcm}$) is calculated and declared as the new RCM point. 

\paragraph{Improving the control matrix $\bm{K}$:} The control matrix $K$ used in this thesis was determined empirically and proved to be a good trade-off between accuracy and convergence speed. However, it is certainly not optimal for every kind of movement within the workspace of the robot inside the eye, as it was adjusted using only the injection procedure presented in Section \ref{Setup}. A better $K$ can be determined by systematically computing accuracy and convergence during movements within the whole workspace inside the eye for different combinations of entries in $K$.

\paragraph{Inclusion of a third rotation into the Jacobian matrix and simulation:} The model of the surgical robot considered in this thesis uses a straight needle as a tool, thus the rotation around the tool axis has no effect on the position of the needle tip and can be neglected. Having established the kinematic relationships in the robot model and a strategy to calculate both the unconstrained and augmented Jacobian matrix, a needle with a curved tip can be implemented. A curved needle allows for injections lateral to the surface and extends the reachable surface area for injection inside the eye, but makes it more difficult to enter the eye through the trocar while maintaining the RCM point. To implement this, the Jacobian matrix needs to be extended by one row expressing the rotation of the tool around its own axis and the augmented Jacobian matrix must be updated accordingly.

\paragraph{Inclusion of dynamic effects into the Simulation:} The V-REP simulation environment is capable of performing precise dynamic calculations. It might be interesting to estimate the loss in precision of both inverse kinematics methods when dynamic effects are considered. 